\documentclass[11pt, a4paper]{article}    
\usepackage[utf8]{inputenc}    
\usepackage[portuguese]{babel}    
\usepackage{indentfirst}    
\usepackage{geometry}    
\geometry{        
 a4paper,    
 top=50mm,    
 }

\title{%
	\vspace{-5cm}
	5 peças para videoconferência
	}
\author{Felippe Brandão}
\date{}
\begin{document}

\maketitle
\section{De um em um}
Frases completas deverão ser formadas, sendo cada participante responsável por somente uma palavra de cada frase. Abra seu microfone somente quando for dizer uma palavra.

\section{Falta de ar programada}
Uma pessoa (aquela que disser primeiro) é incarregada de abrir e fechar o microfone das demais. O microfone de quem estiver aberto deve ser soprado inimterruptamente.

\section{Escuta profunda}
Todos os microfones ficam abertos e ninguém fala, só se escuta.

\section{Querido diário}
Todas as pessoas recontam, ao mesmo tempo, tudo o que realizaram no dia até então.

\section{Monólogos interrompidos}
Cada participante tenta realizar um monólogo até que outra pessoa interrompa o discurso.
Tópicos sugeridos:
\begin{enumerate}
	\item Sons etc. 
	\item A alta do dólar
	\item Seu filme favorito
	\item Teorias da conspiração
	\item Singularidade tecnológica 
\end{enumerate}

\end{document}
